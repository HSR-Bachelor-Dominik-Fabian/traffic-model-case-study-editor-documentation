\section{Vorgehen}
\subsection{Projektmanagement}
\subsubsection{Aufbau}
\begin{itemize}
	\item Elaboration: 
	\item Construction: 
	\item Transition: 
\end{itemize}
\subsubsection{Tools}
Für das Projektmanagement werden folgende Tools verwendet:
\begin{itemize}
\item JIRA: für die Planung und Verwaltung der Arbeitspakete.
\item GitHub: Der Code und die Dokumentation werden über Github Repositories verwaltet
\item CI Server (JetBrains TeamCity): Mittels eines CI Server wird die Qualität und Lauffähigkeit des Codes geprüft
\item TexMaker: Latex Editor für die Dokumentation
\item IntelliJ IDEA 15: IDE für Java Entwicklung
\end{itemize}
\subsection{Entwicklung}
\subsubsection{Vorgehen}
Die Entwicklung der Bachelorarbeit wird als Agiles Softwareprojekt aufgebaut.
\subsubsection{Unit Testing}
\subsubsection{Code Reviews}
Code Reviews und Pair Programming sind Methoden um die Qualität des Codes innerhalb des Teams hoch zu halten
\subsubsection{Code Analyse}
\subsubsection{Code Style Guidelines}