\chapter{Abstract}
Das Unternehmen Senozon AG ist immer wieder mit Fragestellungen wie \glqq{}Was passiert, wenn diese Strasse gesperrt wird?\grqq{} oder \glqq{}Was passiert, wenn diese Strasse eine andere Geschwindigkeitsbeschränkung hätte?\grqq{} konfrontiert. Um diese Fragestellungen zu beantworten unterhält die Senozon AG ein Modell, das sowohl das Strassennetz als auch privater und öffentlicher Verkehr beinhaltet. Mit Hilfe des OpenSource Simulations Framework MATSim kann dieses Modell simuliert werden.\\
Die Arbeit \glqq{}Verkehrsmodell-Fallstudien-Editor\grqq{} behandelt die Entwicklung und Implementation einer Web Applikation, die das interaktive Bearbeiten von Daten eines solchen Verkehrsmodells innerhalb einer Karte ermöglicht. Die grosse Herausforderung dabei, ist vor allem der effiziente Umgang mit der grossen Datenmenge. Dafür wird der QuadTile Algorithmus von OpenStreetMap \cite{OSMQuadTiles} eingesetzt. Weitere Optimierungen auf Software-, wie auch Datenbankseite sind ein wichtiger Teil dieser Arbeit.\\
Die fertige Implementation stellt eine performante Web Applikation dar, die im Design der Senozon AG erscheint. Änderungen an Strassen können vorgenommen und in einer Datenbank persistiert werden. Dazu wurde ein Konzept entwickelt, welches ermöglicht die Änderungen abzuspeichern ohne das Stammnetzwerk von Senozon anzupassen.\\