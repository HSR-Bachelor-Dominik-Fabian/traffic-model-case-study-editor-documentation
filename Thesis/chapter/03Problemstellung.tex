\chapter{Problemstellung}
Immer wieder kommen Kunden der Senozon AG mit Fragestellungen wie \glqq{} Was passiert, wenn diese Strasse gesperrt wird?\grqq{} oder \glqq{} Was passiert, wenn da ein Neubaugebiet mit 5000 Einwohnern gebaut wird?\grqq{}. Bislang sind diese Fragestellungen mit viel Handarbeit verbunden, um das Modell anzupassen. Dies insbesondere, weil die Simulationsdaten von MATSim XML basiert und meist sehr gross sind.\\
Zur Zeit kann lediglich die Senozon AG selbst die Änderungen an den Verkehrsmodellen vornehmen. Durch das Entwickeln einer Web Applikation, die den Kunden ebenfalls zur Verfügung steht, kann nun neu auch der Kunde eine Verkehrsfallstudie erstellen und dann selbstständig simulieren lassen.
\section{Problem dateibasierte Simulation}
Die Simulationssoftware MATSim arbeitet mit dateibasierten Simulationdaten. D.h. die benötigten Informationen für die Simulation, werden in XML Dateien zur Verfügung gestellt. Diese XML Dateien können bis zu mehreren GigaBytes gross werden, wenn es sich um grössere Netzwerke, wie z.B. das Netz der gesamten Schweiz, handelt.\\
Für die Bearbeitung dieser Daten verwendet die Senozon AG zur Zeit eine Client Applikation, die direkt mit den XML Dateien arbeitet. Dadurch ist diese Applikation sehr Arbeitsspeicher intensiv und für die Installation beim Kunden nicht optimal.
\section{Änderungsmanagement}
Die Änderungen, die in der Client Applikation gemacht werden, überschreiben die vorhandenen Daten in der XML Datei. Dadurch muss für jede Verkehrsfallstudie der komplette Datenstamm kopiert und dadurch redundant abgelegt werden. Sollte nun ein Update der vorhandenen Stammdaten vorliegen, müsste dieses Backup bei allen Studien eingepflegt werden.\\