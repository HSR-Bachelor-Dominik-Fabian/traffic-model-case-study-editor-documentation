\chapter{Anforderungen \& Risiken} \label{ch:anforderungen_section}
\section{Anforderungen}
Im Rahmen dieser Bachelorarbeit wurden folgende funktionalen und nicht funktionalen Anforderungen von der Senozon AG an die Arbeit gestellt.
\subsection{Funktionale Anforderungen}
Der entwickelte Prototyp soll insbesondere aufzeigen, wie und mit welchen Konzepten gewisse Funktionalitäten dieser Web Applikation implementiert werden könnten. Dafür wurden folgende funktionalen Anforderungen an das Frontend sowie das Backend gestellt:
\begin{itemize}
\itemsep0em
\item Die Applikation soll das aktuelle Verkehrsmodell mit Hilfe einer Karte darstellen.
\item Der Benutzer hat die Möglichkeit, ein Change Set zu laden, löschen und erstellen.
\item Der Benutzer kann ein Attribut einer Strasse bearbeiten und die Änderung persistent speichern.
\item Der Benutzer hat die Möglichkeit, über ein Formular neue Stammdaten zu importieren.
\item Der Benutzer kann die Richtung einer Strasse ändern.
\end{itemize}
\subsection{Nicht-funktionale Anforderungen}
\label{ch:NFRs}
Zusätzlich zu den funktionalen Anforderungen ist es bei diesen grossen Datenmengen sehr wichtig, die nicht-funktionalen Anforderungen zu erfüllen. Folgende nicht-funktionalen Anforderungen haben uns vor besondere Herausforderungen gestellt:
\begin{itemize}
\itemsep0em
\item Performance: Effiziente Kommunikation zwischen Frontend und Backend für die Übertragung von Netzwerkdaten.
\item Performance: Eine Abfrage inkl. Filterung von Netzwerkdaten auf der Datenbank muss unter einer Sekunde beantwortet werden können.
\item Bedienbarkeit: Das User Interface soll einfach und möglichst ohne Tutorial bedienbar sein.
\item Aussehen: Das User Interface soll im Design von der Senozon AG erscheinen.
\item Skalierbarkeit: Die Web Applikation soll skalierbar sein, damit auch grössere Mengen an Daten bewältigt werden können.
\end{itemize}
\section{Risiken}
In einer ersten Phase der Bachelorarbeit wurde eine Analyse der bestehenden Risiken in diesem Projekt durchgeführt. Während der gesamten Arbeit war das primäre Ziel, diese Risiken vollständig zu eliminieren.
\subsection*{Risiko 1: Grosse Datenmenge}
Durch die grosse Datenmenge besteht die Gefahr, dass sehr viel Zeit in die Performance Optimierung investiert werden muss. Sollte dieses Risiko eintreffen, wird es sehr schwierig werden, einen voll funktionsfähigen Editor zu entwickeln.\\
Eine Massnahme, um dieses Risiko möglichst klein zu halten, ist es, die grosse Datenmenge bereits bei der Evaluation der Konzepte des Projektes zu berücksichtigen. Zusätzlich ist es sehr wichtig, möglichst früh die Performance des Prototyps mit grossen Datenmengen zu testen. Dadurch kann sehr schnell eingeschätzt werden, ob dieses Risiko weiterhin besteht oder bereits durch dafür vorgesehene Konzepte beseitigt werden konnte.\\
Dieses Risiko ist nicht eingetreten. Durch den Einsatz von verschiedenen Konzepten, wie z.B. des QuadTile Algorithmus \cite{OSMQuadTiles} oder verschiedenen Optimierungen auf Seite der Software und der Datenbank, konnte eine gute bis sehr gute Performance erreicht werden.
\subsection*{Risiko 2: Aufwändige Backend Implementierung}
Die Konzeptionierung und Implementation dieser Applikation konzentriert sich hauptsächlich auf das User Interface. Das Backend soll stark vereinfacht angenommen werden. Jedoch besteht das Risiko, dass ohne komplexeres Backend die grossen Datenmengen auf dem User Interface nicht effizient dargestellt werden können.\\
Als Massnahme wurde möglichst früh mit grossen Datenmengen gearbeitet, sodass der Umfang der Optimierungen, die auf der Seite des Backends vorgenommen werden müssen, zum Vorschein kam. Es stelle sich heraus, dass dieser Umfang  um einiges grösser war als erwartet. Dies bedeutete einen deutlich höheren Aufwand in der Implementation des Backends, denn ohne der effizienten Darstellung der Daten ist der Nutzen dieser Web Applikation nicht sehr gross.