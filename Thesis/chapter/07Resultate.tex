\chapter{Resultate}
\section{Performance}
TODO
\section{Rückblick Technologien}
\subsection*{Leaflet}
Für die Darstellung der Karte wird die Library Leaflet, in Verbindung mit D3.js, verwendet. Die Vorteile von Leaflet liegen in der Möglichkeit verschiedene Layer dynamisch zu erzeugen und übereinander anzuordnen. Leaflet besitzt auch die Möglichkeit GeoJSON zu interpretieren und in SVG Geometrien darzustellen. Nachteile sind die fehlende Touch-Optimierung, durch welche es schwer ist die Karte auf Mobile-Geräten zu verwenden. Weiter besitzt Leaflet keine native Möglichkeit GeoJSON Daten von einem Service über das Tile-System zu beziehen. Diese Möglichkeit musste mit dem Plugin Leaflet GeoJSON Tile Layer hinzugefügt werden.
\subsection*{Leaflet GeoJSON Tile Layer}
Dieses Plugin für Leaflet wird eingesetzt um mit dem Tile System GeoJSON Daten vom Service zu beziehen. Diese Daten werden dann von Leaflet interpretiert und als SVG Elemente gerendert. Die Stärke des Plugin ist das einfache Handling. Schwächen besitzt das Plugin in der Implementation, welche nicht in allen belangen perfekt ist und manuell nachgebessert werden musste (z.B. wurden bei Http 204 No Content, Daten erwartet, was dem Standard widerspricht). Das Plugin musste auch mit einer Duplikaten-Erkennung ausgestattet werden, da sonst Strassen mehrfach gezeichnet wurden.
\subsection*{AngularJS}
AngularJS wurde verwendet um die Daten der Formulare, sowie alle Menüs zu verwalten. Die Grösse von Angular ist eine seiner Stärken. Die vielen Funktionen, welche in einem Framework kombiniert sind, arbeiten gut miteinander. Durch die Modularisierung kann der Code sehr gut aufgeteilt werden. Jedoch bringt die Modularisierung auch Schwächen mit sich. Module besitzen einen Scope. Dieser beinhaltet Daten und Methoden welche erreichbar sind. Damit zwei Module zusammenarbeiten können, muss der Scope geteilt werden. Es ist jedoch nicht immer klar, welches Modul mit welchem Scope zur Zeit erreichbar ist. Dadurch ist die Fehlersuche zum Teil ziemlich schwierig.
\subsection*{Play Framework}
Das Play Framework bildet die Basis des SimMapEditors. Im Play Framework sind die Views eingetragen, welche die Single Page App aufbauen. Die Stärke von Play ist der eingebaute Netty Server. Dieser erlaubt ein einfaches Deployment, da die Software nicht bei jedem Build in ein Apache geladen werden muss, sondern direkt gestartet werden kann. Über funktionale Stärken / Schwächen kann keine fundierte Aussage gemacht werden, da für die Implementation des SimMapEditor vor allem Javascript verwendet wurde.
\subsection*{Dropwizard}
Dropwizard ist ein Paket aus verschiedenen Libraries zur Implementation von REST Services. Das Paket beinhaltet JAX-RS mit Jersey für die REST Funktionalitäten. Für das Erstellen von JSON wird Jackson eingebaut, sowie ein NettyServer für das Deployment. Dropwizard erlaubt es einem ohne grossen Aufwand ein lauffähigen und einfach deployable Webservice zu erstellen. Jersey und JAX-RS sind starke Libraries, welches einfach ermöglichen einen REST Service (Maturity Level 2) zu implementieren.
