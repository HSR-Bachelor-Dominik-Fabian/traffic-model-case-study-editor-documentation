\chapter{Resultate}
\section{Performance}
Wie in Kapitel \ref{ch:NFRs} \nameref{ch:NFRs} beschrieben, ist die Performance eine der wichtigsten nicht-funktionalen Anforderungen. Dies hatte zur Folge, dass besonders viel Zeit in deren Optimierung investiert wurde. In Kapitel \ref{ch:performance} \nameref{ch:performance} wird genauer auf die durchgeführten Optimierungen eingegangen.\\
Grundsätzlich gab es drei verschiedene Stufen von Performance, die erreicht wurden. In diesem Kapitel werden diese drei Stufen genauer analysiert und beschrieben.
\subsubsection*{Stufe 1}
In der ersten Stufe, zu Beginn des Projektes, wurde das gesamte Verkehrsmodell-Netzwerk mittels einer einzigen Abfrage beim Öffnen der Applikation geladen. Dabei wurde mit dem öffentlichen Netzwerk von Santiago de Chile gearbeitet. Dies beinhaltet ca. 60'000 Datensätze. Gemäss Tests benötigte diese Abfrage im Durchschnitt ca. 6 Sekunden. Mittels dieser Vorgehensweise wäre es unmöglich gewesen, ein Netzwerk wie die Schweiz mit ca. 4 Millionen Datensätzen in dieser Applikation zu laden und anzuzeigen.
\subsubsection*{Stufe 2}
In der nächsten Stufe wurde der QuadTile Algorithmus von OpenStreetMap eingeführt. Dadurch wird nicht mehr das gesamte Verkehrsmodell-Netzwerk geladen, sondern die Daten in Bereiche aufgeteilt. In dieser Stufe war es bereits möglich mit den 4 Millionen Datensätzen aus dem Verkehrsmodell der Schweiz zu arbeiten. Obwohl die Antwortzeiten immer noch deutlich zu hoch waren, konnte das Anzeigen des Schweizer Verkehrsmodells gemeistert werden.\\
Folgende Zugriffszeiten wurden gemessen:\\[0.3cm]
\begin{tabular}{ l c l}
Zoomstufen 10 - 14 & => & 1 - 3 Sekunden \\
Zoomstufen 14 - 18 & => & 3 - 10 Sekunden \\ 
\end{tabular} 
\subsubsection*{Stufe 3}
In der dritten Stufe wurden besonders im Backend sehr viele Optimierungen vorgenommen. Sei dies das Client-seitige Caching, das Verhindern vom erneuten Zeichnen einer Strasse beim Wechsel der Zoomstufe, der Einsatz einer speziellen Operator Klasse für den Index oder die absichtliche Redundanz in dem Datenmodell. Genauere Beschreibungen der einzelnen Optimierungen sind in Kapitel \ref{ch:performance} \nameref{ch:performance} zu finden. Alle diese Optimierungen zusammen führten dazu, dass die Applikation eine deutlich höhere Performance aufwies und dadurch die Bearbeitung des Schweizer Verkehrsmodells kein Problem mehr darstellte.\\
Folgende Zugriffszeiten wurden gemessen:\\[0.3cm]
\begin{tabular}{ l c l}
Zoomstufen 10 - 14 & => & 50 - 100 Millisekunden \\
Zoomstufen 14 - 18 & => & 100 - 300 Millisekunden \\ 
\end{tabular}\\
\subsubsection*{Resultat}
Die daraus resultierte Performance erfüllt die Anforderungen und ermöglicht ein benutzerfreundliches Bearbeiten des Schweizer Verkehrsmodells. Dennoch entsprechen die geladenen Links und Nodes einer sehr grosse Datenmenge, zum Teil mehr als 10'000 Datensätze pro Anzeige. Das Rendering dieser Datenmenge benötigt eine gewisse Zeit und verzögert dadurch das Darstellen der Daten auf der Karte. Auf die Geschwindigkeit des Renderings des Browsers konnte jedoch kein Einfluss genommen werden.
\newpage
\section{Rückblick Technologien}
\subsection*{Leaflet}
Für die Darstellung der Karte wird die Library Leaflet in Verbindung mit d3.js verwendet. Ein Vorteil von Leaflet liegt darin, verschiedene Schichten dynamisch zu erzeugen und übereinander anzuordnen. Zusätzlich können Daten im GeoJSON Format interpretiert werden und als SVG Geometrien dargestellt werden. Ein Nachteil ist die fehlende Optimierung für Touchscreens wodurch die Verwendung auf Mobilen Geräten erschwert wird. Zudem fehlt Leaflet die Möglichkeit GeoJSON Daten in Form des Tile-Systems zu beziehen. Diese Funktionalität wurde dann durch das Plugin Leaflet GeoJSON Tile Layer hinzugefügt.
\subsection*{Leaflet GeoJSON Tile Layer}
Leaflet GeoJSON Tile Layer ist ein Plugin für Leaflet, das die Möglichkeit bietet Daten in Form des Tile-Systems von einem Service zu beziehen. Diese Daten werden anschliessend von Leaflet interpretiert und als SVG Elemente gerendert. Eine klare Stärke dieses Plugins ist die einfache Verwendung. Dieses Plugin besitzt auch einige Schwächen in der Implementation. Beispielsweise werden bei einer Antwort mit dem HTTP Status Code 204 - No content Daten erwartet was jedoch dem Standard widerspricht. Hierfür und für das Erkennen von Duplikate wurde das Plugin angepasst.
\subsection*{AngularJS}
AngularJS wurde verwendet um die Daten der Formulare, sowie alle Menüs zu verwalten. Eine grosse Stärke dieses Frameworks ist die umfangreiche Funktionalität. Eine dieser Funktionen ist die Modularisierung, die das Aufteilen des Codes in Module ermöglicht. Jedoch bringt diese Modularisierung auch Schwierigkeiten mit sich. Jedes Modul beinhaltet Methoden und Daten die sich im Scope des jeweiligen Moduls befinden. Damit zwei Module interagieren können muss der Scope geteilt werden. Diese Umsetzung war aufgrund der fehlenden oder mangelhaften Dokumentation sehr umständlich.
\subsection*{Play Framework}
Das Play Framework bildet die Basis des SimMapEditors. Damit wurden die Views der Single Page Applikation erstellt. Eine Stärke des Play Frameworks ist sicherlich der eingebaute Netty Server der ein einfaches Deployment erlaubt. Dadurch kann die Applikation direkt als eigenständige Web Applikation gestartet werden und muss nicht in einen Web Server geladen werden. Über die funktionalen Stärken und Schwächen kann keine fundierte Aussage getroffen werden, da für die Logik hauptsächlich Javascript verwendet wurde.
\subsection*{Dropwizard}
Dropwizard stellt ein Werkzeugkoffer mit verschiedenen Libraries zur Implementation von REST Service zur Verfügung. Zum einen enthält sie JAX-RS mit Jersey für die REST Funktionalitäten, JSON Jackson für die Bearbeitung von JSON Objekten und ein Netty Server für das eigenständige Ausführen der Applikation. Eine grosse Stärke ist sicherlich die einfache Anwendung. Ohne grossen Aufwand kann ein lauffähiger Webservice erstellt und deployed werden. Zudem ermöglichen die beiden Libraries JAX-RS in Verbindung mit Jersey die Implementation eines REST Services gemäss den Richardson Maturity Levels. Im Rahmen dieses Projektes wurde z.B. der Maturity Level 2 umgesetzt.
\subsection*{Allgemein}
Im Rahmen dieses Projektes wurde folgende Technologien verwendet:\\
\begin{figure}[H]
\centering
\includegraphics[height=8cm]{images/cloud.png}
\caption{Technologien Cloud}
\label{fig:technologycloud}
\end{figure}
\noindent