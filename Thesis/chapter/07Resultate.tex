\chapter{Resultate}
\section{Performance}
Wie in Kapitel \ref{ch:NFRs} \nameref{ch:NFRs} beschrieben, ist die Performance eine der wichtigsten nicht-funktionalen Anforderungen. Dies hatte zur Folge, dass besonders viel Zeit in dessen Optimierung investiert wurde. In Kapitel \ref{ch:performance} \nameref{ch:performance} wird genauer auf die durchgeführten Optimierungen eingegangen.\\
Grundsätzlich gab es drei verschiedene Stufen von Performance, die erreicht wurden. In diesem Kapitel werden diese vier Stufen genauer analysiert und beschrieben.
\subsubsection*{Stufe 1}
In der ersten Stufe, zu Beginn des Projektes, wurde das gesamte Verkehrsmodell-Netzwerk mittels einer einzigen Abfrage beim Öffnen der Applikation geladen. Dabei wurde mit dem öffentlichen Netzwerk von Santiago de Chile gearbeitet. Dies beinhaltet ca. 60'000 Datensätze. Gemäss Tests benötigte diese Abfrage im Durchschnitt ca. 6 Sekunden. Mittels dieser Vorgehensweise wäre es unmöglich gewesen, ein Netzwerk wie die Schweiz, mit ca. 4 Millionen Datensätzen, auf dieser Applikation zu laden und anzuzeigen.
\subsubsection*{Stufe 2}
In der nächsten Stufe 
\subsubsection*{Stufe 3}
\subsubsection*{Stufe 4}
v0.2:
Arbeitete noch mit Santiago, Lädt das komplette JSON mit ca. 60'000 Datensätzen
Benötigt für das Laden dieser Datensätze:
5954 ms
5717 ms
5951 ms
5978 ms
6175 ms
6184 ms
6204 ms
5924 ms
6058 ms
5882 ms
-------
60'027 ms / 10 = 6002,7 ms

Browser würde mit 4 Millionen Datensätzen komplett crashen.

v0.3:
Vorher:
Bei tiefen Zoomlevel:
1000ms - 3000ms
Bei höheren Zoomlevel:
3000ms - 10000ms

Nach dem entfernen von JOINS und Hinzufügen von Index
Bei tiefen Zoomlevel:
50ms - 100ms
Bei höheren Zoomlevel:
50ms - 300ms

v0.4:
Performance deutlich höher:
bei tiefen Zoomlevel:
50ms - 100ms
bie höheren Zoomlevel:
50ms - 300ms

\section{Rückblick Technologien}
\subsection*{Leaflet}
\subsection*{Leaflet GeoJSON Tile Layer}
\subsection*{AngularJS}
\subsection*{Play Framework}
\subsection*{Dropwizard}