
\chapter{Schlussfolgerung}
\section{Erkenntnisse}
Die Arbeit \glqq{}Verkehrsmodell-Fallstudien-Editor\grqq{} war eine sehr umfangreiche Arbeit mit vielen interessanten Herausforderungen. Sie beinhaltete die Bearbeitung von vielen verschiedenen neuen Themengebieten, in denen das Projektteam noch nicht sehr viel Erfahrung hatte. Sei dies die Kartendarstellung im Browser, das Anzeigen von Strassen auf dieser Karte oder der effiziente Umgang mit grossen Datenmengen.\\
Der Fokus dieser Arbeit lag zu Beginn auf dem Frontend. Das Backend durfte vereinfacht angenommen werden. Schnell wurde jedoch klar, dass ohne komplexeres Backend, das effiziente Darstellen des Frontends, nicht möglich ist. Aufgrund dessen, wurde entschieden, dass trotzdem mehr Zeit in das Backend investiert wird. Diese Entscheidung erst ermöglichte das Entwickeln eines performanten Verkehrsmodell-Fallstudien-Editors.\\
Das Backend vereinfacht anzunehmen machte zu Beginn Sinn. Jedoch wurde bereits bei erste Tests ersichtlich, dass das Frontend Performanceprobleme hat. Dies, obwohl erst mit einer mittleren Datenmenge (ca. 30'000 Datensätze) gearbeitet wurde. Das vereinfachte Backend stellte den kompletten Datenstamm eines Verkehrsmodells für das Frontend zur Verfügung. Dies war dann bei einer grösseren Datenmenge wie z.B. der Schweiz (ca. 4 Millionen Datensätze) undenkbar. Daher wurde der Ansatz des QuadTile Algorithmus von OpenStreetMap \cite{OSMQuadTiles} verwendet, der die Bereiche einer Karte aufteilt und eindeutig identifizierbar macht. Die Verwendung dieses Algorithmus setzt eine gewisse Komplexität des Backends voraus. Der Mehraufwand dieser Komplexität, führte zu einer Veränderung des Funktionsumfangs dieser Arbeit. Das Feature \glqq{}Neue Strasse zeichnen\grqq{} konnte auf Grund dieser Veränderung nur vereinfacht umgesetzt werden.\\
Zudem war das Rendern der unterschiedlichen Browser ein wichtiges Thema. Google Chrome kan besser mit grossen Datenmengen umgehen, als der Mozilla Firefox oder Microsoft Edge Browser. Damit aber die Web Applikation auf allen Browsern genutzt werden kann, wurden Tests auf allen Browser durchgeführt und mit Hilfe des QuadTile Algorithmus die Datenmenge für das Rendern so klein wie möglich gehalten.\\
\section{Weiterführung der Arbeit}
TODO
\section{Persönliches Fazit}
\subsection{Dominik Heeb}
TODO
\subsection{Fabian Keller}
TODO