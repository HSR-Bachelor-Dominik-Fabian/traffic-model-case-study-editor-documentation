\chapter{Schlussfolgerung}
\section{Erkenntnisse}
Die Arbeit \glqq{}Verkehrsmodell-Fallstudien-Editor\grqq{} war eine sehr umfangreiche Arbeit mit vielen interessanten Herausforderungen. Sie beinhaltete die Bearbeitung von vielen verschiedenen neuen Themengebieten, in denen das Projektteam noch nicht sehr viel Erfahrung hatte. Dazu gehörten beispielsweise die Kartendarstellung im Browser, das Anzeigen von Strassen auf dieser Karte oder der effiziente Umgang mit grossen Datenmengen.\\
Der Fokus dieser Arbeit lag zu Beginn auf dem Frontend. Das Backend durfte vereinfacht angenommen werden. Schnell wurde jedoch klar, dass ohne komplexeres Backend das effiziente Darstellen des Frontends nicht möglich ist. Aufgrund dessen wurde entschieden, dass trotzdem mehr Zeit in das Backend investiert wird. Diese Entscheidung erst ermöglichte das Entwickeln eines performanten Verkehrsmodell-Fallstudien-Editors.\\
Das Backend vereinfacht anzunehmen machte zu Beginn Sinn. Jedoch wurde bereits bei ersten Tests ersichtlich, dass das Frontend Performanceprobleme aufwies, dies obwohl erst mit einer mittleren Datenmenge (ca. 60'000 Datensätze) gearbeitet wurde. Das vereinfachte Backend stellte den kompletten Datenstamm eines Verkehrsmodells für das Frontend zur Verfügung. Dies war dann bei einer grösseren Datenmenge wie z.B. der Schweiz (ca. 4 Millionen Datensätze) undenkbar. Daher wurde der Ansatz des QuadTile Algorithmus von OpenStreetMap \cite{OSMQuadTiles} verwendet, der die Bereiche einer Karte aufteilt und eindeutig identifizierbar macht. Die Verwendung dieses Algorithmus setzt eine gewisse Komplexität des Backends voraus. Der Mehraufwand dieser Komplexität führte zu einer Veränderung des Funktionsumfangs dieser Arbeit. Das Feature \glqq{}Neue Strasse zeichnen\grqq{} konnte auf Grund dieser Veränderung nur vereinfacht umgesetzt werden.\\
Zudem war das Rendern der unterschiedlichen Browser ein wichtiges Thema.
Google Chrome kann besser mit grossen Datenmengen umgehen als z.B. Mozilla Firefox oder der Microsoft Edge Browser. Der Unterschied war zu Beginn in der Benutzung der Applikation deutlich zu sehen. Damit aber die Web Applikation auf allen Browsern genutzt werden kann, wurden Tests auf allen Browser durchgeführt und mit Hilfe des QuadTile Algorithmus die Datenmenge für das Rendern so klein wie möglich gehalten.\\
\section{Backlog}
Der Umfang dieser Arbeit beschränkte sich aus zeitlichen Gründen auf einen gewissen Umfang. Folgende Themen können bei einer Weiterentwicklung dieses Projektes noch umgesetzt werden.
\subsubsection*{Netzwerk-ID automatisch hochzählen}
Zur Zeit ist es lediglich möglich, ein einziges Verkehrsmodell zu importieren. Im Rahmen dieses Projektes reichte dies vollkommen. Wird ein zusätzliches Modell importiert, wird für dieses Netzwerk wieder die ID 1 vergeben und alle Daten zu einem grossen Netzwerk zusammengeschlossen. Dies stellt kein Problem dar, solange sich die IDs der Nodes und Links der beiden Netzwerke unterscheiden.
\subsubsection*{Changeset basierend auf bestehendem Changeset}
Zur Zeit kann ein leeres Changeset erstellt oder ein bestehendes Changeset bearbeitet werden, jedoch nicht ein neues Changeset basierend auf einem bestehendem Changeset erstellt werden. Dafür müsste lediglich das Changeset mit allen Link\textunderscore Changes und Node\textunderscore Changes kopiert werden und eine neue ID vergeben werden. Dies wäre eine nützliche Funktion für die Weiterentwicklung dieses Projekts.
\subsubsection*{Export Funktion von XML}
Die Daten können über ein Formular in der Web Applikation importiert werden. Jedoch wurde die Funktionalität, die bearbeiteten Daten wieder zu exportieren, noch nicht implementiert. Das Datenmodell wurde so vorbereitet, dass alle Daten noch vorhanden sind und für den Export verwendet werden können.
\subsubsection*{Neue Strasse / Node zeichen}
Das Zeichnen einer neuen Strasse oder eines neuen Nodes war zu Beginn ebenfalls Teil dieser Arbeit. Jedoch musste aus zeitlichen Gründen darauf verzichtet werden. Das Datenmodell ist auch hierfür vorbereitet. Um dies in Zukunft zu implementieren muss berücksichtigt werden, dass zur Zeit für jede empfangene Strasse geprüft wird, ob ein Eintrag im Changeset vorhanden ist. Ist dies der Fall, wird der Datensatz aus dem Changeset verwendet. Ansonsten wird der normale Datensatz gezeichnet. Eine neue Strasse ist jedoch noch nicht in den Stammdaten vorhanden und wird somit nie geprüft. Dadurch würde eine neue Strasse mit der jetzigen Logik nie gezeichnet werden. Diese Logik müsste auf der Seite des Clients so angepasst werden, dass neue Strassen im Changeset am Ende noch zusätzlich gezeichnet werden.
\section{Persönliches Fazit}
\subsection{Dominik Heeb}
Die Bachelorarbeit Verkehrsmodell-Fallstudien-Editor war von Anfang bis Schluss eine sehr spannende und vielseitige Arbeit. Das Themengebiet über die Implementation von Maps mit interaktiven Daten, war für uns komplettes Neuland. Im Team konnten wir uns jedoch sehr schnell in das Thema einleben. Die Implementation eines REST Services, welcher in einer skalierbaren Infrastruktur gehalten wird, war eines der Highlights der Arbeit für mich. Dabei konnte ich das erste Mal in einem Projekt mit Docker arbeiten, was für mich eine sehr gute und spannende Technologie darstellt. Weiter war es überaus spannend mit Datenmengen im Millionenbereich zu arbeiten und die Probleme kennenzulernen welche diese mit sich bringen. Das Team, eingeschlossen der Projektpartner Senozon AG, hat sehr gut harmoniert. Durch eine enge und koordinierte Zusammenarbeit, konnten alle Hauptrisiken in kürzester Zeit bewältigt werden.
Persönlich konnte ich sehr viel von diesem Projekt für meine Zukunft profitieren.
\subsection{Fabian Keller}
Die Bachelorarbeit \glqq{}Verkehrsmodell-Fallstudien-Editor\grqq{} war für mich eine sehr positive Erfahrung. Zu Beginn war die Thematik für beide Teammitgleider ziemlich neu und daher eine grosse Herausforderung. Dazu gehörten beispielsweise die Kartendarstellung im Browser, das Anzeigen von Strassen auf dieser Karte oder der effiziente Umgang mit grossen Datenmengen. Jedoch konnte wir uns durch grossen Einsatz und hohe Motivation ziemlich schnell in die Thematik einarbeiten und die grössten Risiken in kurzer Zeit beseitigen. Ein Highlight dieser Arbeit war sicherlich die erreichte Performance in der Darstellung der grossen Datenmenge auf der Karte. Zudem gefällt mir das User Interface Konzept der Web Applikation sehr. Durch die Vielseitigkeit dieser Arbeit konnte ich persönlich sehr viel profitieren und werde sicherlich in meiner zukünftigen Tätigkeit als Software Engineer einiges wiederverwenden können. Die Zusammenarbeit im Team sowie auch die Zusammenarbeit mit dem Projektpartner Senozon AG war einwandfrei und hat mir sehr viel Spass gemacht.